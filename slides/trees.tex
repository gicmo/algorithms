% THE BEER-WARE LICENSE
% adrian.stoewer@rz.ifi.lmu.de wrote this file. As long as you retain this notice
% you can do whatever you want with this stuff. If we meet some day, and you think
% this stuff is worth it, you can buy me a beer in return.


\documentclass{beamer}

% Standard packages

\usepackage[english]{babel}
\usepackage[latin1]{inputenc}
\usepackage{times}
\usepackage[T1]{fontenc}

% Theme listings

\usepackage{minted}
\usemintedstyle{native}
\newminted{cpp}{
    gobble=0,
    linenos,
    samepage=true,
    frame=none,
    fontsize=\scriptsize
}

% Theme slides

\usetheme{Warsaw}
\setbeamercolor{normal text}{fg=white,bg=black!90}
\setbeamercolor{structure}{fg=white}
\setbeamercolor{alerted text}{fg=red!85!black}
\setbeamercolor{item projected}{use=item,fg=black,bg=item.fg!35}
\setbeamercolor*{palette primary}{use=structure,fg=structure.fg}
\setbeamercolor*{palette secondary}{use=structure,fg=structure.fg!95!black}
\setbeamercolor*{palette tertiary}{use=structure,fg=structure.fg!90!black}
\setbeamercolor*{palette quaternary}{use=structure,fg=structure.fg!95!black,bg=black!80}
\setbeamercolor*{framesubtitle}{fg=white}
\setbeamercolor*{block title}{parent=structure,bg=black!60}
\setbeamercolor*{block body}{fg=black,bg=black!10}
\setbeamercolor*{block title alerted}{parent=alerted text,bg=black!15}
\setbeamercolor*{block title example}{parent=example text,bg=black!15}

%% Info

\title[BST and Heaps] 
{
    Introduction to tree data structures with binary search tree 
    and heap as examples
}

\author[Adrian Stoewer]{
    Adrian~Stoewer
}

%% START
\begin{document}

\begin{frame}
  \titlepage
\end{frame}

%\begin{frame}{Outline}
%  \tableofcontents
%\end{frame}


%% Analysis of Algorithms
\begin{frame}
    \begin{center}
        \begin{LARGE}
            Analysis of Algorithms
        \end{LARGE}
    \end{center}
\end{frame}

\begin{frame}{Worst case analysis}
    \begin{itemize}
    
        \item Many algorithms run close to their worst case.
        
        \item The best case is often not very significant, because many
              algorithms perform equally well in their best case.
              
        \item Determine the mean performance is often not trivial.
        
        \item The worst case defines the upper boundary. If the resource consumption
              is acceptable for the worst case this applies to all cases.
              
        \item But there are cases where it makes more sense to analyse the mean
              performance.
    
    \end{itemize}
\end{frame}


\begin{frame}{Big O notation}
    \begin{itemize}
    
        \item The big O notation defines the upper bound of the resource
              consumption of an algorithm depending on the size of the input $n$.
              
        \item $g(n)$ is the upper bound of a function $f(n)$ if there exists a constant
              $c$ and $n_0$ such that $f(n) \leq cg(n)$ is true for every $n > n_0$.
    
    \end{itemize}
\end{frame}


\begin{frame}{Simple rules for the big O notation}
    \begin{itemize}
    
        \item Dont care about constants: If $T(n) = n + 50$ is the runtime of an algorithm
              the constant factor for $n = 1024$ is less than 5\%.
              
        \item Don't care about constant factors: For every input larger than 10 
              $T_1(n) = n^2$ is already worse than $T_1(n) = 10n$.
              
        \item Sums are calculated using the maximum.
    
    \end{itemize}
\end{frame}


\begin{frame}{Code and the big O notation}
    \begin{itemize}
    
        \item Ignore everything that is not a loop where the number of cycles depends
              on the input.
              
        \item Ignore everything that is not a recursion where the depth depends on 
              the input.
              
    \end{itemize}
\end{frame}


\begin{frame}[fragile]{Code and the big O notation}
\begin{cppcode}
int a[n];

for (int i = 0; i < n; i++)     // O(?)
    a[i] = random();

for (int i = 1; i < n; i++) {   // O(?)
    int tmp = a[i];
    int j = i;
    while (j > 1 && a[j] > tmp) {
        a[j] = a[j - 1];
        j--;
    }
    a[j] = tmp;
}
\end{cppcode}
\end{frame}


%% Trees as data structures
\begin{frame}
    \begin{center}
        \begin{LARGE}
            Trees as data structures
        \end{LARGE}
    \end{center}
\end{frame}


%% Binary search tree (BST)
\begin{frame}
    \begin{center}
        \begin{LARGE}
            Binary search tree (BST)
        \end{LARGE}
    \end{center}
\end{frame}


\begin{frame}{What is a BST}
    \begin{itemize}
        \item Each node has a left child, right child and some value.
        \item Each value is only contained once (represents a set).
        \item The value of the left child is always smaller than the parents value.
        \item The value of the right child is always larger than the parents value. 
    \end{itemize}
\end{frame}


\begin{frame}[fragile]{Basic BST node structure}
\begin{cppcode}
struct bst {
    int value;

    bst *parent;
    bst *left;
    bst *right;

    bst(int value, bst *parent=nullptr, bst *left=nullptr, 
        bst *right=nullptr);
};
\end{cppcode}
\end{frame}


\begin{frame}[fragile]{Add node to BST}
\begin{cppcode}
void bst_add(bst *&tree, int value, bst *parent=nullptr);


void bst_add(bst *&tree, int value, bst *parent) {
    if (!tree)
        tree = new bst(value, parent);
    else if (value < tree->value)
        bst_add(tree->left, value, tree);
    else if (value > tree->value)
        bst_add(tree->right, value, tree);
}
\end{cppcode}
\end{frame}


\begin{frame}[fragile]{Remove node from BST}
\begin{cppcode*}{fontsize=\tiny}
bool bst_remove(bst *&tree, int value) {
    if (!tree)
        return false;
    else if (value < tree->value)
        return bst_remove(tree->left, value);
    else if (value > tree->value)
        return bst_remove(tree->right, value);

    bst *tmp = tree;
    if (!tree->left && !tree->right) {
        tree = nullptr;
    } else if (tree->left && !tree->right) {
        tree->left->parent = tree->parent;
        tree = tree->left;
    } else if (!tree->left && tree->right) {
        tree->right->parent = tree->parent;
        tree = tree->right;
    } else {
        bst *tmp_right = tree->right;
        tree->left->parent = tree->parent;
        tree = tree->left;
        bst_add_node(tree, tmp_right, tree->parent);
    }
    delete tmp;

    return true;
}
\end{cppcode*}
\end{frame}

%% Heaps
\begin{frame}
    \begin{center}
        \begin{LARGE}
            Heaps (BST)
        \end{LARGE}
    \end{center}
\end{frame}

\begin{frame}[fragile]{What is a Heap?}
TODO
\end{frame}


\end{document}
